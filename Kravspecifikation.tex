\documentclass[11pt]{article}
\usepackage[utf8]{inputenc}  % To control and create table of content
\usepackage{fancyhdr} 	% To create header
\usepackage{dirtytalk} % To create citations
\usepackage{array} % To control and create fixed size tables

\pagestyle{fancy}
\fancyhf{}
\lhead{Kravspecifikation}
\rhead{Version 0.1}
\rfoot{Page \thepage}

\renewcommand*\contentsname{Indholdsfortegnelse}

\begin{document}
	\begin{titlepage}
		\begin{center}
			\Large\textbf{Kravspecifikation}\\
			\large\textit{Version: 0.1}
		\end{center}
	\end{titlepage}
	
	\tableofcontents
	\newpage
	
	\section{Introduktion og baggrund}
	Formålet med dette dokument er at beskrive funktionelle og ikke-funktionelle krav for per konditionerings blodtryks apparatet. De funktionelle krav vil blive beskrevet ved hjælp af fully dressed use case diagrammer. En lille ændring
	
	Beskrivelse af projektet: 
	
	\say{\textit{Akut blodprop i hjernen (Acute Ischemic Stroke – AIS) er en førende årsag til død og alvorlig handicap hos personer over 60 år.Intravenøs trombolysebehandling administreret indenfor 4,5 time fra symptomdebut er den nuværende bedste medicinske behandling. Grundet sikkerhedshensyn og det snævre tidsvindue er det desværre kun et fåtal af AIS patienterne, der modtager denne behandling. Målet er at opløse blodproppen og genoprette blodforsyning og dermed redde hjernevæv, der lider af iltmangel men endnu ikke er dødt. Om et område af hjernen dør eller står til at redde ved en blodprop afhænger ikke kun af selve blodproppen men også om hjernen er i stand til at få blod via omveje dannet af hjernens små blodkar. Et område af hjernen går til grunde med det samme (infarktkernen). Denne kerne af dødt hjernevæv kan i dagene efter en blodprop sprede sig og vokse. Der er således behov for at kunne beskytte hjernen mod iltmangel og øge andelen af hjernevæv, der overlever en blodprop. Iltmangel induceret periodevis i et fjernt organ (remote ischemic conditioning RIC) kan udføres ved at puste en blodtryksmanchet med afklemning af armen. Konditionering kan leveres som pre, per, og postconditionering, afhængig af om stimulus udøves før iltmangel, under iltmangel men før blodproppen er opløst og endelig efter blodproppen er opløst. Dyrestudier og senest kliniske studier har vist at RIC kan mindske det område af hjertet eller hjernen, der dør ved en blodprop. Det er ikke tilstrækkeligt undersøgt om RIC mindsker risikoen for handicap efter en blodprop i hjernen.}}
	
	\section{System beskrivelse}
	System er beregnet til behandling af patienter med \textit{acute ischemic stroke(AIS)}. Formålet er pre, per og postkonditioning af disse patienter. Systemet skal kunne lave arteriel okklusion i de øverste ydre ekstremiteter. For at sikre tilstrækkelig okklusion, skal det systoliske blodtryk først måles og derefter pumpe cuffen op til plus 25 mmHg over det målet tryk. Som minimum skal der afklemmes med et tryk på 180 mmHg. Okklusionen bliver holdt konstant i 5 minutter, hvor efter trykket lukkes ud der holdes en "pause" på 5 minutter. Denne process gentages ind til konditioneringen er færdig. 

	\section{Funktionelle krav}
	
	%Aktør beskrivelse
		\subsection{Aktør beskrivelse}
	\begin{center}
		\begin{tabular}{ | m{4cm} | m{8cm}| } 
			\hline
			Aktørnavn& Patient \\ 
			\hline
			Aktørtype & Primær og sekundær \\ 
			\hline
			Beskrivelse af aktør & En person med AIS som skal konditioneres\\ 
			\hline
		\end{tabular}
	\end{center}
	
	\begin{center}
		\begin{tabular}{ | m{4cm} | m{8cm}| } 
			\hline
			Aktørnavn& Medicinsk personale / Bruger \\ 
			\hline
			Aktørtype & Primær og sekundær \\ 
			\hline
			Beskrivelse af aktør & Aktør som påmontere cuff og styre konditionering eller person som observere de gemte data fra behandlingsforløb\\ 
			\hline
		\end{tabular}
	\end{center}
	
	\begin{center}
		\begin{tabular}{ | m{4cm} | m{8cm}| } 
			\hline
			Aktørnavn& Database \\ 
			\hline
			Aktørtype & Sekundær \\ 
			\hline
			Beskrivelse af aktør & Gemmer data og logs information omkring konditionerings forløb\\ 
			\hline
		\end{tabular}
	\end{center}
	
	%Use case 1
	\subsection{Use case 1 - Konditionering}
\begin{center}
		\begin{longtable}{ | m{4cm} | m{8cm}| } 
			\hline
			Goal& Gennemføre konditioneringsbehandling  \\ 
			\hline
			Initiation &  Medicinsk personale\\
			\hline
			Actors and stakeholders & 
			\begin{itemize}
				\item Medicinsk personale(primær)
				\item Patient (sekundær)
			\end{itemize} \\ 
			\hline
			References & Use case 3 \\ 
			\hline
			Number of concurrent occurrences & En til mange\\ 
			\hline	
			Precondition & 
			\begin{itemize}
				\item Mode switch er sat til “\textit{Konditionering}”
			\end{itemize} \\ 
			\hline
			Postcondition & 
			\begin{itemize}
				\item 5 hele cyklus er gennemført og gemt på hukommelsen
			\end{itemize} \\ 
			\hline
			Main scenario & \begin{enumerate}
				\setlength\itemsep{0cm} % Decrease line distance
				\item \textit{Medicinsk personale} placerer manchetten på patienten
				\item Knappen [Start/Stop] trykkes
				\item Et nyt patient ID genereres
				\subitem[Extension \#1] 
				\item Patient ID’et vises på skærmen
				\item Blodtrykket måles via \textit{use case 3}
				\subitem[Extension \#2]
				\item Blodtrykket vises på displayet og værdien gemmes i hukommelse
				\item Cuffen fyldes med luft til et tryk på 25 mmHG over systolisk tryk (minimum 180 mmHg)
			\end{enumerate} \\ 
			\hline
			Main scenario & \begin{enumerate}
				\setlength\itemsep{0cm} % Decrease line distance
				\setcounter{enumi}{7}				
				\item Tidsstempel gemmes når trykket er opnået
				\item Trykket opretholdes i 5 minutter(Okklusion) og resterende tid vises på displayet
				\item Blodtrykket måles via \textit{use case 3}
				fra punkt 2.
				\item Deflaterer cuffen helt og forbliver i dette stadie i 5 min(Reperfusion) Ved deflation start gemmes tidsstempel. Tid til næste okklusion vises på displayet
				\item Gentag punkt 7-11 (en cyklus) fire gange. Det nuværende cyklus nummer vises i displayet
			\end{enumerate} \\ 
			\hline
			Extensions & [Extension \#1] Et patient ID eksisterer allerede på apparatet. Der genereres ikke noget nyt patient ID.
			
			[Extension \#2] Blodtrykket kunne ikke måles. Gentag use case 3 hvis extension 2 ikke lige er eksekveret. Ellers skrives i display “FEJL kunne ikke måle blodtryk” og use casen stopper.  \\
			\hline
		\end{longtable}
		
	\end{center}
	\pagebreak


	
	%Use case 2
		\subsection{Use case 2 - Konditionering cyklusser - ikke færdig}
		\begin{center}
			\begin{tabular}{ | m{4cm} | m{8cm}| } 
				\hline
				Goal& Gennemfør én cyklusser\\ 
				\hline
				Initiation &  Medicinsk personale\\
				\hline
				Actors and stakeholders & 
				\begin{itemize}
					\item Medicinsk personale(Primær)
					\item Patient (Sekundær)
				\end{itemize} \\ 
				\hline
				References & - \\ 
				\hline
				Number of concurrent occurrences & Én til mange\\ 
				\hline	
				Precondition & 
				\begin{itemize}
					\item Det systoliske blodtryk er kendt
				\end{itemize} \\ 
				\hline
				Postcondition & 
				\begin{itemize}
					\item 5 cyklusser er blevet gennemført 
				\end{itemize} \\ 
				\hline
				Main scenario & \begin{enumerate}
					\item Manchetten blæses op til 200 mmHg 
					\item Manchetten pumpes op til trykket overstiger det systoliske tryk
					\item Luften lukkes gradvis ud af manchetten(Specificer hastigheden) indtil der registreres oscillationer og det aktuelle tryk logges. 
					\item Det systoliske blodtryk vises på displayet
				\end{enumerate} \\ 
				\hline
				Extensions &  -\\ 
				\hline
			\end{tabular}
		\end{center}
	
	%Use case 3
		\subsection{Use case 3 - Mål blodtryk}
		\begin{center}
			\begin{tabular}{ | m{4cm} | m{8cm}| } 
				\hline
				Goal& Mål et systolisk, diastolisk og middel(MAP) tryk\\ 
				\hline
				Initiation &  Use case 1 eller 2\\
				\hline
				Actors and stakeholders & - \\
				\hline
				References & - \\ 
				\hline
				Number of concurrent occurrences & En til mange\\ 
				\hline	
				Precondition & 
				\begin{itemize}
					\item Patient ID er oprettet
					\item Manchetten er placeret på armen
					\item Mode switch er sat til “\textit{Konditionering}”
				\end{itemize} \\ 
				\hline
				Postcondition & 
				\begin{itemize}
					\item Blodtrykket er mål
				\end{itemize} \\ 
				\hline
				Main scenario & \begin{enumerate}
					\setlength\itemsep{0cm} % Decrease line distance
					\item Manchetten fyldes til tryk over systolisk niveau 
					\item Luften lukkes gradvist ud og det systoliske tryk registreres 
					\item Middel trykket (MAP) måles 
					\item Det diastoliske tryk udregnes ud fra systole og MAP 
					\item Blodtrykket vises på skærmen og værdien gemmes i hukommelsen med et tidsstempel 
				\end{enumerate} \\ 
				\hline
				Extensions & -\\ 
				\hline
			\end{tabular}
		\end{center}
	
	%Use case 4
		\subsection{Use case 4 - Hent gemt data}
		\begin{center}
			\begin{tabular}{ | m{4cm} | m{8cm}| } 
				\hline
				Goal& Eksportér data fra blodtryksapparat til databasen\\ 
				\hline
				Initiation &  Medicinsk personale\\
				\hline
				Actors and stakeholders & 
				\begin{itemize}
					\item Medicinsk personale(primær)
					\item Patient (sekundær)
				\end{itemize} \\ 
				\hline
				References & - \\ 
				\hline
				Number of concurrent occurrences & Én pr behandlingsforløb \\ 
				\hline	
				Precondition & 
				\begin{itemize}
					\item Der eksisterer en logfil på hukommelsen
				\end{itemize} \\ 
				\hline
				Postcondition & 
				\begin{itemize}
					\item Logfile overført til database
					\item Blodtryksapparat udstyres med formateret hukommelse og klar til næste patient
				\end{itemize} \\ 
				\hline
				Main scenario & \begin{enumerate}
					\setlength\itemsep{0cm} % Decrease line distance
					\item Tag SD kortet ud af blodtryksapparatet 
					\item Sæt SD kortet i computeren og overfør filen 
					\item Formatér SD kortet
					\item Sæt SD kortet tilbage i konditioneringsapparatet 
				\end{enumerate} \\ 
				\hline
				Extensions & -\\ 
				\hline
			\end{tabular}
		\end{center}
	
	%UseCase 5
		\subsection{Use case 5 - Okklusionstræning}
		\begin{center}
			\begin{tabular}{ | m{4cm} | m{8cm}| } 
				\hline
				Goal& Gennemføre okklusionstræning\\ 
				\hline
				Initiation &  Patient\\
				\hline
				Actors and stakeholders & 
				\begin{itemize}
					\item Patient (Primær)
					\item Medicinsk personale(Sekundær)
				\end{itemize} \\ 
				\hline
				References & - \\ 
				\hline
				Number of concurrent occurrences & Én pr træningspas \\ 
				\hline	
				Precondition & 
				\begin{itemize}
					\item Ingen
				\end{itemize} \\ 
				\hline
				Postcondition & 
				\begin{itemize}
					\item Okklusionssæt gennemfør
				\end{itemize} \\ 
				\hline
				Main scenario & \begin{enumerate}
					\item Montere manchetten på arm/ben
					\item Tryk på knap [Okklusionstræning]
					\item Manchetten pumpes op til 100mmHg
					\item Træningssættet begyndes og trykkes holdes konstant på 100mmHg (+/-10mmHg)
					\item Tryk på knap [Stop]
				\end{enumerate} \\ 
				\hline
				Extensions & -\\ 
				\hline
			\end{tabular}
		\end{center}
	
\end{document}