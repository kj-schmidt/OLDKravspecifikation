\documentclass[11pt]{article}
\usepackage[utf8]{inputenc}  % To control and create table of content
\usepackage{fancyhdr} 	% To create header
\usepackage{dirtytalk} % To create citations
\usepackage{array} % To control and create fixed size tables

\pagestyle{fancy}
\fancyhf{}
\lhead{Kravspecifikation}
\rhead{Version 0.1}
\rfoot{Page \thepage}

\renewcommand*\contentsname{Indholdsfortegnelse}

\begin{document}
	\begin{titlepage}
		\begin{center}
			\Large\textbf{Kravspecifikation}\\
			\large\textit{Version: 0.1}
		\end{center}
	\end{titlepage}
	
	\tableofcontents
	\newpage
	
	\section{Introduktion og baggrund}
	Formålet med dette dokument er at beskrive funktionelle og ikke-funktionelle krav for per konditionerings blodtryks apparatet. De funktionelle krav vil blive beskrevet ved hjælp af fully dressed use case diagrammer.
	
	Beskrivelse af projektet: 
	
	\say{\textit{Akut blodprop i hjernen (Acute Ischemic Stroke – AIS) er en førende årsag til død og alvorlig handicap hos personer over 60 år.Intravenøs trombolysebehandling administreret indenfor 4,5 time fra symptomdebut er den nuværende bedste medicinske behandling. Grundet sikkerhedshensyn og det snævre tidsvindue er det desværre kun et fåtal af AIS patienterne, der modtager denne behandling. Målet er at opløse blodproppen og genoprette blodforsyning og dermed redde hjernevæv, der lider af iltmangel men endnu ikke er dødt. Om et område af hjernen dør eller står til at redde ved en blodprop afhænger ikke kun af selve blodproppen men også om hjernen er i stand til at få blod via omveje dannet af hjernens små blodkar. Et område af hjernen går til grunde med det samme (infarktkernen). Denne kerne af dødt hjernevæv kan i dagene efter en blodprop sprede sig og vokse. Der er således behov for at kunne beskytte hjernen mod iltmangel og øge andelen af hjernevæv, der overlever en blodprop. Iltmangel induceret periodevis i et fjernt organ (remote ischemic conditioning RIC) kan udføres ved at puste en blodtryksmanchet med afklemning af armen. Konditionering kan leveres som pre, per, og postconditionering, afhængig af om stimulus udøves før iltmangel, under iltmangel men før blodproppen er opløst og endelig efter blodproppen er opløst. Dyrestudier og senest kliniske studier har vist at RIC kan mindske det område af hjertet eller hjernen, der dør ved en blodprop. Det er ikke tilstrækkeligt undersøgt om RIC mindsker risikoen for handicap efter en blodprop i hjernen.}}
	
	\section{System beskrivelse}
	System er beregnet til behandling af patienter med \textit{acute ischemic stroke(AIS)}. Formålet er pre, per og postkonditioning af disse patienter. Systemet skal kunne lave arteriel okklusion i de øverste ydre ekstremiteter. For at sikre tilstrækkelig okklusion, skal det systoliske blodtryk først måles og derefter pumpe cuffen op til plus 25 mmHg over det målet tryk. Som minimum skal der afklemmes med et tryk på 180 mmHg. Okklusionen bliver holdt konstant i 5 minutter, hvor efter trykket lukkes ud der holdes en "pause" på 5 minutter. Denne process gentages ind til konditioneringen er færdig. 

	\section{Funktionelle krav}
	
	%Aktør beskrivelse
		\subsection{Aktør beskrivelse}
	\begin{center}
		\begin{tabular}{ | m{4cm} | m{8cm}| } 
			\hline
			Aktørnavn& Patient \\ 
			\hline
			Aktørtype & Primær og sekundær \\ 
			\hline
			Beskrivelse af aktør & En person med AIS som skal konditioneres\\ 
			\hline
		\end{tabular}
	\end{center}
	
	\begin{center}
		\begin{tabular}{ | m{4cm} | m{8cm}| } 
			\hline
			Aktørnavn& Medicinsk personale / Bruger \\ 
			\hline
			Aktørtype & Primær og sekundær \\ 
			\hline
			Beskrivelse af aktør & Aktør som påmontere cuff og styre konditionering eller person som observere de gemte data fra behandlingsforløb\\ 
			\hline
		\end{tabular}
	\end{center}
	
	\begin{center}
		\begin{tabular}{ | m{4cm} | m{8cm}| } 
			\hline
			Aktørnavn& Database \\ 
			\hline
			Aktørtype & Sekundær \\ 
			\hline
			Beskrivelse af aktør & Gemmer data og logs information omkring konditionerings forløb\\ 
			\hline
		\end{tabular}
	\end{center}
	
	%Use case 1
	\subsection{Use case 1 - Måling af systolisk blodtryk}
\begin{center}
		\begin{tabular}{ | m{4cm} | m{8cm}| } 
			\hline
			Goal& Måling af systolisk blodtryk \\ 
			\hline
			Initiation &  Medicinsk personale\\
			\hline
			Actors and stakeholders & 
			\begin{itemize}
				\item Medicinsk personale(Primær)
				\item Patient (Sekundær)
			\end{itemize} \\ 
			\hline
			References & - \\ 
			\hline
			Number of concurrent occurrences & Én per behandling\\ 
			\hline	
			Precondition & 
			\begin{itemize}
				\item Patienten er registreret med et unikt ID
				\item Blodtryksmanchetten er monteret på patientet
			\end{itemize} \\ 
			\hline
			Postcondition & 
			\begin{itemize}
				\item The systoliske blodtryk er målt og konditionering kan begynde
			\end{itemize} \\ 
			\hline
			Main scenario & \begin{enumerate}
				\item Personalet monterer manchetten
				\item Manchetten pumpes op til trykket overstiger det systoliske tryk
				\item Luften lukkes gradsvis ud af manchetten(Specificer hastigheden) indtil der registreres oscillationer og det aktuelle tryk logges. 
				\item Det systoliske blodtryk vises på displayet
			\end{enumerate} \\ 
			\hline
			Extensions & [Extension 1] Det systoliske blodtryk kunne ikke måles: Gentag use case fra punkt 2 \\ 
			\hline
		\end{tabular}
	\end{center}
	
	%Use case 2
		\subsection{Use case 2 - Konditionering cyklusser - ikke færdig}
		\begin{center}
			\begin{tabular}{ | m{4cm} | m{8cm}| } 
				\hline
				Goal& Gennemfør én cyklusser\\ 
				\hline
				Initiation &  Medicinsk personale\\
				\hline
				Actors and stakeholders & 
				\begin{itemize}
					\item Medicinsk personale(Primær)
					\item Patient (Sekundær)
				\end{itemize} \\ 
				\hline
				References & - \\ 
				\hline
				Number of concurrent occurrences & Én til mange\\ 
				\hline	
				Precondition & 
				\begin{itemize}
					\item Det systoliske blodtryk er kendt
				\end{itemize} \\ 
				\hline
				Postcondition & 
				\begin{itemize}
					\item 5 cyklusser er blevet gennemført 
				\end{itemize} \\ 
				\hline
				Main scenario & \begin{enumerate}
					\item Manchetten blæses op til 200 mmHg 
					\item Manchetten pumpes op til trykket overstiger det systoliske tryk
					\item Luften lukkes gradvis ud af manchetten(Specificer hastigheden) indtil der registreres oscillationer og det aktuelle tryk logges. 
					\item Det systoliske blodtryk vises på displayet
				\end{enumerate} \\ 
				\hline
				Extensions &  -\\ 
				\hline
			\end{tabular}
		\end{center}
	
	%Use case 3
		\subsection{Use case 3 - Hent data og reset}
		\begin{center}
			\begin{tabular}{ | m{4cm} | m{8cm}| } 
				\hline
				Goal& Eksporter data gemt på blodtryk\\ 
				\hline
				Initiation &  Medicinsk personale\\
				\hline
				Actors and stakeholders & 
				\begin{itemize}
					\item Medicinsk personale(Primær)
					\item Database (Sekundær)
				\end{itemize} \\ 
				\hline
				References & - \\ 
				\hline
				Number of concurrent occurrences & Én pr. blodtryksapparat\\ 
				\hline	
				Precondition & 
				\begin{itemize}
					\item Blodtryksapparatet er blevet returneret efter brug
					\item Data er blevet gemt på apparatet
				\end{itemize} \\ 
				\hline
				Postcondition & 
				\begin{itemize}
					\item Se oversigt over behandlingsforløb og cyklusser 
					\item Apparatet er reset og klar til brug
				\end{itemize} \\ 
				\hline
				Main scenario & \begin{enumerate}
					\item Forbind apparatet til computeren
					\item Eksporter data
					\item Formatér drevet på apparatet 
				\end{enumerate} \\ 
				\hline
				Extensions & -\\ 
				\hline
			\end{tabular}
		\end{center}
	
	%Use case 4
		\subsection{Use case 4 - Sikkerhedskontrol med pulsoximeter}
		\begin{center}
			\begin{tabular}{ | m{4cm} | m{8cm}| } 
				\hline
				Goal& Sikre at patientens kredsløb tåler konditionering\\ 
				\hline
				Initiation &  Medicinsk personale\\
				\hline
				Actors and stakeholders & 
				\begin{itemize}
					\item Medicinsk personale(Primær)
					\item Patient (Sekundær)
				\end{itemize} \\ 
				\hline
				References & - \\ 
				\hline
				Number of concurrent occurrences & Én pr behandlingsforløb \\ 
				\hline	
				Precondition & 
				\begin{itemize}
					\item Pulsoximeteret er monteret på patients finger
					\item Patient har gennemført én afklemnings cyklus
				\end{itemize} \\ 
				\hline
				Postcondition & 
				\begin{itemize}
					\item Patientens tilstand er bestemt 
				\end{itemize} \\ 
				\hline
				Main scenario & \begin{enumerate}
					\item Efter først afklemnings cyklus måles saturation og puls
					\item Der detekteres en puls
					\item Behandlingen kan fortsætte
				\end{enumerate} \\ 
				\hline
				Extensions & [Extension 1] Tegn på dårlig kredsløb: Blodtryksapparatet stopper konditionerings forløbet\\ 
				\hline
			\end{tabular}
		\end{center}
	
	%UseCase 5
		\subsection{Use case 5 - Okklusionstræning}
		\begin{center}
			\begin{tabular}{ | m{4cm} | m{8cm}| } 
				\hline
				Goal& Gennemføre okklusionstræning\\ 
				\hline
				Initiation &  Patient\\
				\hline
				Actors and stakeholders & 
				\begin{itemize}
					\item Patient (Primær)
					\item Medicinsk personale(Sekundær)
				\end{itemize} \\ 
				\hline
				References & - \\ 
				\hline
				Number of concurrent occurrences & Én pr træningspas \\ 
				\hline	
				Precondition & 
				\begin{itemize}
					\item Ingen
				\end{itemize} \\ 
				\hline
				Postcondition & 
				\begin{itemize}
					\item Okklusionssæt gennemfør
				\end{itemize} \\ 
				\hline
				Main scenario & \begin{enumerate}
					\item Montere manchetten på arm/ben
					\item Tryk på knap [Okklusionstræning]
					\item Manchetten pumpes op til 100mmHg
					\item Træningssættet begyndes og trykkes holdes konstant på 100mmHg (+/-10mmHg)
					\item Tryk på knap [Stop]
				\end{enumerate} \\ 
				\hline
				Extensions & -\\ 
				\hline
			\end{tabular}
		\end{center}
	
\end{document}